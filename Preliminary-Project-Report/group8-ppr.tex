% CVPR 2023 Paper Template
% based on the CVPR template provided by Ming-Ming Cheng (https://github.com/MCG-NKU/CVPR_Template)
% modified and extended by Stefan Roth (stefan.roth@NOSPAMtu-darmstadt.de)

\documentclass[10pt,twocolumn,letterpaper]{article}

%%%%%%%%% PAPER TYPE  - PLEASE UPDATE FOR FINAL VERSION
% \usepackage[review]{cvpr}      % To produce the REVIEW version
\usepackage{cvpr}              % To produce the CAMERA-READY version
%\usepackage[pagenumbers]{cvpr} % To force page numbers, e.g. for an arXiv version

% Include other packages here, before hyperref.
\usepackage{graphicx}
\usepackage{amsmath}
\usepackage{amssymb}
\usepackage{booktabs}


% It is strongly recommended to use hyperref, especially for the review version.
% hyperref with option pagebackref eases the reviewers' job.
% Please disable hyperref *only* if you encounter grave issues, e.g. with the
% file validation for the camera-ready version.
%
% If you comment hyperref and then uncomment it, you should delete
% ReviewTempalte.aux before re-running LaTeX.
% (Or just hit 'q' on the first LaTeX run, let it finish, and you
%  should be clear).
\usepackage[pagebackref,breaklinks,colorlinks]{hyperref}


% Support for easy cross-referencing
\usepackage[capitalize]{cleveref}
\crefname{section}{Sec.}{Secs.}
\Crefname{section}{Section}{Sections}
\Crefname{table}{Table}{Tables}
\crefname{table}{Tab.}{Tabs.}


%%%%%%%%% PAPER ID  - PLEASE UPDATE
\def\cvprPaperID{*****} % *** Enter the CVPR Paper ID here
\def\confName{CVPR}
\def\confYear{2023}


\begin{document}

%%%%%%%%% TITLE - PLEASE UPDATE
\title{Super-Resolution for Enhancing Classification Accuracy on Medical Images}

% Use one of these two author templates. Either works
% The first is a custom template, and not the actual CVPR one
\author{Aditya Somasundaram \qquad Sumanth N R \qquad Ruthwika Boyapally \qquad Gautham Bellamkonda\\
\textit{Indian Institute of Technology Hyderabad, India}\\
{\tt\small \{ee20btech11002, ma20btech11016, ma20btech11004, cs20btech11017\}@iith.ac.in}
}

% \author{Aditya Somasundaram\\
% {\tt\small ee20btech11002}
% \and
% Sumanth N R\\
% {\tt\small ma20btech11016}
% \and
% Ruthwika Boyapally\\
% {\tt\small ma20btech11004}
% \and
% Gautham Bellamkonda\\
% {\tt\small cs20btech11017}
% }
\maketitle

%%%%%%%%% ABSTRACT
\begin{abstract}
   Super-Resolution (SR) techniques have shown promise in enhancing the spatial quality and detail of medical images, addressing issues arising from various imaging constraints, such as limited equipment capabilities, low radiation doses, compressed data transmission, and constrained imaging time. However, its potential impact on the overall quality of medical image analysis and diagnosis remains largely unexplored. We aim to explore how SR affects the identification of critical diagnostic features, with a particular emphasis on its potential to enhance classification accuracy in medical image analysis.
\end{abstract}

%%%%%%%%% BODY TEXT
\section{Introduction}
\label{sec:intro}

Super-Resolution (SR) refers to the process of enhancing the resolution or level of detail in an image or video beyond its original resolution. This technique is widely used in various fields, including image processing, computer vision, and medical imaging. There are two primary approaches to super-resolution.

\subsection{Single-Image Super-Resolution (SISR)} In SISR, the goal is to increase the resolution of a single low-resolution image. This is a challenging task as the missing high-frequency details need to be inferred. Deep learning techniques, particularly Convolutional Neural Networks (CNNs), have shown remarkable success in SISR. These networks are trained on pairs of low and high-resolution images to learn the mapping function that can upscale the low-resolution image to a higher resolution while preserving or generating fine details.

\subsection{Multi-Image Super-Resolution (MISR)}
In MISR, multiple low-resolution images of the same scene are used to generate a high-resolution output. This approach takes advantage of the redundancy of information present in multiple images of the same scene, typically captured from slightly different perspectives or with small shifts. Techniques like image registration and fusion are employed to combine these images into a single, high-resolution result.

\subsection{Applications}
\begin{enumerate}
    \item Image and Video Enhancement: Super-resolution can improve the visual quality of images and videos, making them clearer and more detailed.
    \item Medical Imaging: In medical imaging, super-resolution can be used to enhance the quality of MRI or CT scans, enabling better diagnosis and treatment planning.
    \item Surveillance and Security: It can help in improving the quality of surveillance camera footage, making it easier to identify individuals and objects.
    \item Satellite and Remote Sensing: Super-resolution is used to enhance the resolution of satellite images, aiding in various applications such as land-use classification and environmental monitoring.
    \item Art Restoration: Super-resolution techniques can be applied to restore and enhance old or damaged artworks and historical documents.
\end{enumerate}

The effectiveness of super-resolution techniques often depends on the quality of the algorithms and the amount of training data available. Deep learning methods, especially Generative Adversarial Networks (GANs), have made significant strides in achieving impressive super-resolution results by learning complex patterns and structures from large datasets. As technology continues to advance, super-resolution techniques are likely to become even more powerful and applicable in various domains.

%-------------------------------------------------------------------------
\section{Problem Statement}
\label{sec:problem}

Super-Resolution (SR) techniques have shown promise in enhancing the spatial quality and detail of medical images, addressing issues arising from various imaging constraints, such as limited equipment capabilities, low radiation doses, compressed data transmission, and constrained imaging time. However, while SR can significantly improve image resolution, its potential impact on the overall quality of medical image analysis and diagnosis remains largely unexplored.

Current approaches primarily focus on the application of SR to medical images with the objective of improving visual clarity and detail. Nevertheless, there exists a critical knowledge gap regarding how SR impacts the diagnostic accuracy and clinical utility of these enhanced medical images. Specifically, it is unclear how SR affects the identification of critical diagnostic features, the reliability of automated image analysis algorithms.

Therefore, there is an imperative need to investigate the broader implications of SR in the context of medical image analysis beyond mere visual enhancement. We aim to comprehensively examine the effects of SR on the accuracy, reliability, and clinical relevance of medical image analysis. Specifically, we aim to investigate whether SR can be used to enhance discriminative diagnostic features, such that the transformed images are more amenable to classification. By addressing this gap, we seek to unlock the full potential of SR techniques in the field of medical imaging, ultimately advancing patient care, diagnosis, and treatment planning.

% Accurate classification of medical images is of paramount importance in modern healthcare, as it directly impacts diagnostic precision, treatment planning, and patient outcomes. However, a persistent challenge in this domain is the accurate classification of low-resolution medical images, which are often encountered due to various imaging constraints, such as limited equipment capabilities, low radiation doses, constrained imaging time, or compressed data transmission.

% Low-resolution medical images inherently suffer from a deficiency in spatial detail and clarity, creating a obstacle for automated classification algorithms in their task of identifying crucial diagnostic features. Consequently, this limitation can lead to misclassifications, potentially compromising patient care and clinical decision-making.


\section{Literature Review}
Dong et al. \cite{dong2015image} were pioneers in utilizing deep learning methods to address super-resolution challenges in the context of natural image datasets. Building on this foundation, Kim et al. \cite{kim2016accurate} introduced very deep convolutional networks to achieve even more accurate image super-resolution. Following this, Lim et al. \cite{lim2017enhanced} extended the application of deep learning with enhanced deep residual networks, emphasizing the role of residual learning in capturing fine image details. Meanwhile, Kim et al. \cite{kim2016deeply} explored the concept of deeply-recursive convolutional networks, incorporating multiple recursive stages within a CNN to progressively enhance image resolution. Further advancements came with Tai et al. \cite{tai2017image} introducing deep recursive residual networks tailored for image super-resolution, providing an innovative approach to reconstructing high-resolution details. Shifting focus to generative adversarial networks (GANs), Ledig et al. \cite{ledig2017photo} presented a GAN-based approach for photo-realistic single image super-resolution, highlighting the potential of GANs in generating high-quality images. In the realm of GANs, Wang et al. \cite{wang2018esrgan} proposed enhanced Super-Resolution Generative Adversarial Networks (ESRGAN), emphasizing the application of GANs for improved super-resolution quality. 

Shifting toward medical image analysis, Mahaprata et al. \cite{mahapatra2019image} \cite{mahapatra2017retinal} applied progressive GANs to enhance the resolution of medical images, expanding the utility of GANs to healthcare contexts and also combined super-resolution with retinal vasculature segmentation using GANs, demonstrating the applicability of GANs in medical image enhancement. In the field of medical diagnostics, Umehara et al. \cite{umehara2018application} applied super-resolution convolutional neural networks to improve image resolution in chest CT scans. Furthermore, Park et al. \cite{park2018computed} explored the use of deep convolutional neural networks to enhance the resolution of computed tomography (CT) images, offering potential benefits for medical diagnostics. Recent advancements include a Nature Scientific Reports article by Ahmad et al. \cite{ahmad_nature} introducing a novel GAN architecture specifically designed for medical image super-resolution, representing cutting-edge developments in the field. 

Collectively, these papers reflect the diverse approaches and methodologies employed in image super-resolution, from deep learning and recursive architectures to the application of GANs, with significant implications for various domains, including healthcare and diagnostics.

%%%%%%%%% REFERENCES
{\small
\bibliographystyle{ieee_fullname}
\bibliography{egbib}
}

\end{document}